\documentclass[pdflatex,12pt,a4paper,twoside]{scrartcl}
\usepackage[a4paper,left=30mm,right=30mm,top=36mm,bottom=36mm]{geometry}
\usepackage{authblk,amsmath,amsfonts,amssymb,amsthm,ngerman}
\usepackage{graphicx}
\usepackage{epsfig}
\usepackage{nicefrac}
\usepackage{mathrsfs}
\usepackage{hyperref}
\usepackage[utf8]{inputenc}
\usepackage[T1]{fontenc}
\begin{document}

\section{Herleitung des Laplace-Modells}

\subsection{Annahmen}

Die Luft ist inkompressibel. Das heißt $\rho$ (Dichte der Luft) ist konstant.
Effekte der Viskosität und Turbulenzen werden vernachlässigt.
Das Geschwindigkeitsfeld der Luftmoleküle ist ein wirbelfreies Potentialfeld.
$\nabla \times v =0$ und $v= \nabla \phi$, wobei $\phi$ eine skalare Funktion ist.

\subsection{Herleitung der Laplace-Gleichung}

Einsetzen dieser Annahme in die Kontinuitätsgleichung
\begin{align*}
\frac{\partial \rho}{\partial t}+\nabla \cdot (\rho v)=0
\end{align*}

und Ausnutzen, dass $\rho$ konstant ist, liefert die Laplace-Gleichung
\begin{align*}
\Delta \phi = \frac{\partial^2 \phi}{\partial x^2}+\frac{\partial^2 \phi}{\partial y^2}+\frac{\partial^2 \phi}{\partial z^2}=0
\end{align*}

\subsubsection{Randbedingungen}
An den Rändern nehmen wir Neumann-Randbedingungen an:
\begin{align*}
n \cdot \nabla \phi =v_N
\end{align*}
Für Wände gilt, dass $v_N=0$ ist. Bei Lüftern variiert dieser Wert.

Aus Gründen der Eindeutigkeit muss $\phi$ an einer Stelle bekannt sein. Hierzu reicht es auf einem Randpunkt gleich Null zu setzen.

\subsection{Wärmekonvektion}

Zur Berechnung unserer Temperatur benötigen wir noch eine Gleichung, die den Wärmeübertrag beschreibt. Diese ist:
\begin{align*}
\rho c_p v \cdot \nabla T - \nabla \cdot (k\nabla T)=0
\end{align*}
T ist hierbei die Temperatur, $\rho$ die Dichte, $c_p$ die spezifische Wärme und $k$ die Wärmeleitfähigkeit von Luft. 

Einsetzen ergibt:
\begin{align*}
\rho c_p \nabla \phi \cdot \nabla T - k\Delta T=0
\end{align*}

\subsubsection{Randbedingungen}
Auch hier benötigen wir als Randbedingung wieder einen bekannten Wert $T_D$ am Rand.

Desweiteren gibt es noch Neumann-Randbedingungen, welche den Wärme-Zufluss charakterisieren. 
\begin{align*}
-n \cdot (k\nabla T)=q
\end{align*} 


\section{Literatur}
[1] : \url{https://www.ima.umn.edu/preprints/pp2014/2434.pdf}

\end{document}