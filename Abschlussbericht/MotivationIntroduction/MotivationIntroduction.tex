\section{Motivation und Problembeschreibung}


\section{Einleitung}


Im Bereich der hardware-orientierten Numerik wurden bereits in den vergangenen Jahren einige Untersuchungen gemacht. So wurde in den Jahren 2010 und 2011 durch Geveler et al. die Lattice-Boltzmann-Methode zur Lösung von Navier-Stokes- und Flachwassergleichungen auf unterschiedlicher Hardware analysiert. Es stellte sich heraus, dass GPUs Multi-Core CPUs mit einem Speedup von bis zu acht überlegen sind, ohne dabei an Genauigkeit der numerischen Lösung einzubü\ss en. 
In \glqq Efficient Finite Element Geometric Multigrid Solvers for Unstructured Grids on GPUs\grqq\, wurde bereits 2011 nach dem Paradigma der hardware-orientierten Numerik ein geometrisches Mehrgitterverfahren mit finiter Elemente Assemblierung entwickelt, welches nur aus einer Reihe von Sparse Matrix-Vektor Multiplikationen besteht. Durch diese Implementierung, welche keinen Leistungsverlust nach sich zog, war es möglich, die Parallelität von Rechenarchitekturen noch besser auszunutzen. Besonders durch die Verwendung von GPUs anstatt von Multi-Core CPUs ergab sich ein durchschnittlicher Speedup von 8.
Geveler et al. betrachtete 2013 in \glqq 
Towards a complete FEM-based simulation toolkit on GPUs: Geometric Multigrid solver\grqq\, die Lösung partieller Differentialgleichungen mit finiten Elementen und Mehrgitterverfahren auf unstrukturierten Gittern. Es wurde gezeigt, dass sich die Laufzeit der Anwendung erheblich verbessern lässt, sobald GPUs anstatt von Multi-Core CPUs benutzt wurden. 
Des Weiteren wurde in \glqq Energy efficieny vs. Performance of the numerical solution of PDEs: An application study on a low-power ARM-based cluster\grqq\, ein Cluster aus 96 ARM Cortex-A9 Dual-Core Prozessoren einer Rechenarchitektur, basierend auf x86-Prozessoren gegenüber gestellt. Dabei wurde die verbrauchte Energie zur Lösung von drei wissenschaftlichen Anwendungen, einem Finite-Elemente Code mit Mehrgitter-Löser, einer strömungsmechanischen Anwendung und einem Code zur Ausbreitung von Schallwellen mit Hilfe der Spektral-Elemente Methode, analysiert. Schließlich wurde gezeigt, dass die verbrauchte Energie zur Lösung erheblich gesenkt werden kann, im Einklang mit einer akzeptablen Erhöhung der Laufzeit. 
In \glqq Porting FEASTFLOW to the Intel Xeon Phi: Lessons Learned\grqq\, zeigte Geveler et al. die Leistungsverbesserung von axpy-Vektor Operationen und Sparse Matrix-Vektor Multiplikationen durch Nutzung des Intel Xeon Phi Koprozessors. Diese Analyse, sowie in \glqq FFF2: Future-proof High Performance Numerical Simulation for CFD with FEASTFLOW (2)\grqq\, durchgeführten Untersuchungen, welche sich mit der Entwicklung von numerischen Methoden zur parallelen Lösung von partiellen Differentialgleichungen für realitätsnahe industrielle und wissenschaftliche Probleme befassen, basierten auf der Software Infrastruktur \glqq FEASTFLOW\grqq. Dieses Paket umfasst Software zur numerischen Lösung der Navier-Stokes-Gleichungen in 2D und 3D. 
