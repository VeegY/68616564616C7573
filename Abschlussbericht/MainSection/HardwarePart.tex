\section{Hardware}
Der zweite wichtige Aspekt beim Aufbau eines unkonventionellen 
Supercomputers ist der Physikalische aufbau des Computers.
Hierbei muss man darauf achten ein Gleichgewicht zwischen den Faktoren: 
Energieverbrauch, Rechengeschwindigkeit und Kühlung zu finden.
Zusätzlich soll unabhängig von der Größe des Clusters eine
einfache Lösung für Monitoring bereitgestellt werden.


\subsection{Rackdesign}
Für ein Jetson-TK1-Cluster mit insulärer Stromversorgung
ist das Design des Racks der entschiedene Punkt um 
auch unter dauerhafter Höchstbelastung eine
stabile Funktion der Rechenknoten zu gewährleisten.
Durch die Verwendung eines Lithium-Ion-Akkus zur Energiespeicherung
und das Risiko eines Kurzschlusses durch Kondensionswasser muss
ein geschickt aufgebautes Rechencluster gewährleißten, dass die Temperatur
der gesamten Anlage in dem stark restringierten Bereich von 
$18^\circ\text{C}~\text{bis}~40^\circ \text{C}$ gehalten wird. 
%Besonders beim Aufbau eines insulären Jetson-TK1-Clusters ist die Kühlung ein
%heikles Thema, da sowohl die Temperatur des Rechenclusters als auch 
%die Temperatur des Energiespeichers in seinem stark restringierten Bereich liegen müssen.




\subsection{Dashbord}
