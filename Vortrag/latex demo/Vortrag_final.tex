\documentclass[12pt]{beamer}

% proper umlauts
\usepackage[utf8]{inputenc}
\usepackage[T1]{fontenc}

% german
\usepackage[ngerman]{babel}

% AMS math for formulae
\usepackage{amsmath}
\usepackage{amssymb}
\usepackage{amsfonts}

% graphics
\usepackage{graphicx}
\usepackage{subfig}

% some other useful packages
%\usepackage{url}
\usepackage{tabularx}


%%%%%%%%%%%%%%%%%%%%%%%%%%%%%%%%%%%%%%%%%%%%%%%%%%%%%%%%%%%%%%%%%%%%%
%%% my shortcuts
%%%%%%%%%%%%%%%%%%%%%%%%%%%%%%%%%%%%%%%%%%%%%%%%%%%%%%%%%%%%%%%%%%%%%
\newcommand{\highlight}[1]{{\color{blue}\bf#1}}
\newcommand{\todo}[1]{{\color{blue}\bf TODO: #1}}



%%%%%%%%%%%%%%%%%%%%%%%%%%%%%%%%%%%%%%%%%%%%%%%%%%%%%%%%%%%%%%%%%%%%%
%%% setup beamer presentation theme
%%%%%%%%%%%%%%%%%%%%%%%%%%%%%%%%%%%%%%%%%%%%%%%%%%%%%%%%%%%%%%%%%%%%%

\mode<presentation>
{
  \usetheme{TUDortmund2}
}

% include intermediate TOCs automatically at each \section
%\AtBeginSection[]
%{
%  \begin{frame}[c]
%    \frametitle{Overview}
%    \tableofcontents[currentsection]
%  \end{frame}
%}

% Suppress navigation symbols
\setbeamertemplate{navigation symbols}{}


%%%%%%%%%%%%%%%%%%%%%%%%%%%%%%%%%%%%%%%%%%%%%%%%%%%%%%%%%%%%%%%%%%%%%
%%% titlepage information
%%%%%%%%%%%%%%%%%%%%%%%%%%%%%%%%%%%%%%%%%%%%%%%%%%%%%%%%%%%%%%%%%%%%%

\title{Studierendenprojekt 2015/2016}
\author{Betreuer: Markus Geveler}
\institute[TU Dortmund]{
Fakultät für Mathematik \\ TU Dortmund \\
}
\date{31.03.2016}



%%%%%%%%%%%%%%%%%%%%%%%%%%%%%%%%%%%%%%%%%%%%%%%%%%%%%%%%%%%%%%%%%%%%%
%%% titlepage
%%%%%%%%%%%%%%%%%%%%%%%%%%%%%%%%%%%%%%%%%%%%%%%%%%%%%%%%%%%%%%%%%%%%%
\begin{document}
% Use non-transparent version of logo for title page
\logo{\centering%
\includegraphics[height=0.5cm]{figures/logo_TUDortmund}%
\hspace*{1em}%
\includegraphics[height=0.5cm]{figures/logo_fakm}%
\hspace*{15em}}
\begin{frame}[c]
  \titlepage
\end{frame}

% no logo from now on, just eats space
\logo{}

%%%%%%%%%%%%%%%%%%%%%%%%%%%%%%%%%%%%%%%%%%%%%%%%%%%%%%%%%%%%%%%%%%%%%
\section{Inhaltsübersicht}
%%%%%%%%%%%%%%%%%%%%%%%%%%%%%%%%%%%%%%%%%%%%%%%%%%%%%%%%%%%%%%%%%%%%%

\begin{frame}[c]
\frametitle{Inhaltsübersicht}
\begin{itemize}
\item Motivation und Zielsetzung \\
\item Hardware \\
\item Problemstellung \\
\item Numerische Methoden \\
\item Performance \\
\item Fazit und Ausblick
\end{itemize}
\end{frame}

%%%%%%%%%%%%%%%%%%%%%%%%%%%%%%%%%%%%%%%%%%%%%%%%%%%%%%%%%%%%%%%%%%%%%
\section{Motivation und Zielsetzung}
%%%%%%%%%%%%%%%%%%%%%%%%%%%%%%%%%%%%%%%%%%%%%%%%%%%%%%%%%%%%%%%%%%%%%

\begin{frame}[c]
\frametitle{Motivation}
\begin{itemize}
\item Nutzung von GPUs für das wissenschaftliche Rechnen, bedingt durch die hohe Parallelisierbarkeit von Matrix-Vektor-Multiplikationen \\
\item Minimierung des Energieverbrauchs \\
\item Nutzung von erneuerbaren Energiequellen (hier: Solareneergie)
\end{itemize}

\end{frame}


\begin{frame}[c]
\frametitle{Zielsetzung}
\begin{itemize}
\item Entwicklung eines Codes, der ein mathematisches Problem mit bestimmter Genauigkeit löst \\
\item dafür: Entwicklung eines Lösers auf der GPU, der dieses Problem elegant löst 
\item Konstruktion der Hardwareanordnung \\
\item Entwicklung eines Dash-Boards für einfacheren Zugriff auf Systeminformationen
\end{itemize}
\highlight{Wichtigstes Kriterium:}
bestmögliche Performance
\end{frame}

%%%%%%%%%%%%%%%%%%%%%%%%%%%%%%%%%%%%%%%%%%%%%%%%%%%%%%%%%%%%%%%%%%%%%
\section{Hardware}
%%%%%%%%%%%%%%%%%%%%%%%%%%%%%%%%%%%%%%%%%%%%%%%%%%%%%%%%%%%%%%%%%%%%%
\begin{frame}[c]
\frametitle{Auswahl der Hardware}
Daten von TEGRA K1 und Motivation zur Auswahl
\end{frame}


\begin{frame}[c]
\frametitle{Weitere Hardwarekomponenten}
Rack, Container, Solaranlage (alles inkl. Bilder)
\end{frame}

\begin{frame}[c]
\frametitle{Anordnung der Hardware}
Nähere Beschreibung des Aufbaus + Blender-Modell
\end{frame}
%%%%%%%%%%%%%%%%%%%%%%%%%%%%%%%%%%%%%%%%%%%%%%%%%%%%%%%%%%%%%%%%%%%%%
\section{Problemstellung}
%%%%%%%%%%%%%%%%%%%%%%%%%%%%%%%%%%%%%%%%%%%%%%%%%%%%%%%%%%%%%%%%%%%%%
\begin{frame}[c]
\frametitle{Problemstellung}
Kopiere Theorie aus GitHub + kürzen
\end{frame}

%%%%%%%%%%%%%%%%%%%%%%%%%%%%%%%%%%%%%%%%%%%%%%%%%%%%%%%%%%%%%%%%%%%%%
\section{Numerische Methoden}
%%%%%%%%%%%%%%%%%%%%%%%%%%%%%%%%%%%%%%%%%%%%%%%%%%%%%%%%%%%%%%%%%%%%%
\begin{frame}[c]
\frametitle{Finite Differenzen-Ansatz}
Ansatz und modifizierte Diffquotienten
\end{frame}

\begin{frame}[c]
\frametitle{Ergebnis für Problem 1}
Hoffentlich zwei Probleme mit graphischer Ausgabe
\end{frame}

\begin{frame}[c]
\frametitle{Ergebnis für Problem 2}
Hoffentlich zwei Probleme mit graphischer Ausgabe
\end{frame}


\begin{frame}[c]
\frametitle{Finite Elemente-Ansatz}
Bisschen Theorie + Code-Aufbau
möglicherweise zweite Folie
\end{frame}

\begin{frame}[c]
\frametitle{Finite Elemente-Ansatz}
Bisschen Theorie + Code-Aufbau
möglicherweise zweite Folie
\end{frame}

\begin{frame}[c]
\frametitle{Ergebnis für Problem 1}
Hoffentlich zwei Probleme mit graphischer Ausgabe
\end{frame}

\begin{frame}[c]
\frametitle{Ergebnis für Problem 2}
Hoffentlich zwei Probleme mit graphischer Ausgabe
\end{frame}


\begin{frame}[c]
\frametitle{Löser GMRES}
Theorie
\end{frame}


\begin{frame}[c]
\frametitle{Löser GMRES}
Umsetzung
\end{frame}
%%%%%%%%%%%%%%%%%%%%%%%%%%%%%%%%%%%%%%%%%%%%%%%%%%%%%%%%%%%%%%%%%%%%%
\section{Performance}
%%%%%%%%%%%%%%%%%%%%%%%%%%%%%%%%%%%%%%%%%%%%%%%%%%%%%%%%%%%%%%%%%%%%%

\begin{frame}[c]
\frametitle{Performance-Messung}
verschiedene Messungen, möglichst graphisch, Vergleich verschiedener Methoden, etc. ca. 5-6 Folien der Übersichtlichkeit geschuldet
\end{frame}

%%%%%%%%%%%%%%%%%%%%%%%%%%%%%%%%%%%%%%%%%%%%%%%%%%%%%%%%%%%%%%%%%%%%%
\section{Fazit und Ausblick}
%%%%%%%%%%%%%%%%%%%%%%%%%%%%%%%%%%%%%%%%%%%%%%%%%%%%%%%%%%%%%%%%%%%%%
\begin{frame}[c]
\frametitle{Fazit}
Vergleich Ergebnis mit Zielsetzung
\end{frame}

\begin{frame}[c]
\frametitle{Ausblick}
Offene Baustellen, Forsetzung durch das nächste Studienprojekt

\end{frame}


\end{document}